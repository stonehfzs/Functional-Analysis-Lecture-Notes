% \documentclass[lang=cn,newtx,10pt,scheme=chinese]{elegantbook}
\documentclass[theorem=false,mathfont=none,openany,sub3section]{easybook}

\usepackage[lang=cn]{eb-elegantbook}
\usepackage{lmodern}
\usepackage{codehigh}
\lstset{moreemph={nofont,thmenv}}

\renewcommand{\rmdefault}{lmr}
\renewcommand{\sfdefault}{lmss}
\renewcommand{\ttdefault}{lmtt}

\title{Functional Analysis}
\subtitle{Lecture Notes}
\author{Stone Sun}
%\institute{Ocean University of China}
\date{\today}
%\version{}
\bioinfo{联系方式}{hefengzhishui@outlook.com}

%\extrainfo{注意:Elegant系列模板自 2023 年 1 月 1 日开始,不再更新和维护!}

\setcounter{tocdepth}{3}

\logo{logo.jpg}
\cover{cover.jpg}

% 本文档命令
% \usepackage{array}
\newcommand{\ccr}[1]{\makecell{{\color{#1}\rule{1cm}{1cm}}}}

% 修改标题页的橙色带
\definecolor{customcolor}{RGB}{32,178,170}
\colorlet{coverlinecolor}{customcolor}
% \usepackage{cprotect}

\addbibresource[location=local]{reference.bib} % 参考文献,不要删除

% 定义\btocgroup和\etocgroup命令把目录风格限制在组内,使其局部生效
\newcommand{\btocgroup}[1][toc]{\addtocontents{#1}{\string\begingroup}}
\newcommand{\etocgroup}[1][toc]{\addtocontents{#1}{\string\endgroup}}
%\SetTocStyle{chapter}{emph}{tocformat+ = \color{black}}
\let\ls\lstinline
\ebhdrset{footnotetype=flush}
\ctexset{
  paragraph/numbering=false,
  paragraph/beforeskip=1explus.2ex
  }
\SetTocStyle{subsubsection}{sub2}{
  tocindent=3.8em,
  tocformat+=\color{blue}
  }
\UseTocStyle{subsubsection}{sub2}{toc}

\begin{document}

\maketitle
\begin{center}
谨以此篇, 献给热爱分析的你.\par
\end{center}
\frontmatter

\begingroup
\renewcommand{\familydefault}{\rmdefault}
\tableofcontents
\endgroup

\newpage
\begin{center}
\Large
\textbf{前言}\par
\end{center}

\hspace{2em}

\begin{flushright}
\text{Stone Sun}\\
\text{\today}
\end{flushright}

\mainmatter

\btocgroup
\UseTocStyle{chapter}{emph}{toc}
\chapter{距离空间}
\etocgroup

\section{距离空间基本概念}

距离空间是泛函分析研究的基础, 也是拓扑学的基础. 本章我们将介绍距离空间的基本概念, 并举一些例子, 以便理解一些重要的定理.\par

\subsection{距离的定义}

\begin{definition}
  设$X$是一个非空点集, 如果存在一个函数$d: X \times X \to \mathbb{R}$, 使得对于任意的$x, y, z \in X$都有以下公理化条件成立:

  \begin{enumerate}
      \item $d(x, y) = 0 \iff x = y$
      \item $d(x, y) = d(y, x)$
      \item $d(x, z) \leq d(x, y) + d(y, z)$
  \end{enumerate}

  那么我们称$d$为$X$上的一个距离, 并称$(X, d)$为一个距离空间, 简记为$X$.\par

\end{definition}

距离空间上显然成立的一个定理是Cauchy-Schwarz不等式:\par

\begin{definition}
  设$\vec{\alpha}= (a_1, a_2, \cdots, a_n)$和$\vec{\beta} = (b_1, b_2, \cdots, b_n)$是$\mathbb{R}^n$中的两个向量, 则有:\par
  \begin{itemize}
    \item $\vec{\alpha} \cdot \vec{\beta} = \sum_{i=1}^n a_i b_i$
    \item $\sum_{i=1}^n a_i b_i \leq \left(\sum_{i=1}^n a_i^2\right)^{1/2} \cdot \left(\sum_{i=1}^n b_i^2\right)^{1/2}$
    \item $\sum_{i=1}^n \left(a_i+b_i\right)^2 \leq \left(\left(\sum_{i=1}^n a_i^2\right)^{1/2} + \left(\sum_{i=1}^n b_i^2\right)^{1/2}\right)^2$
  \end{itemize}
\end{definition}

\subsection{距离空间的例子}

\begin{example}
    实数集$\mathbb{R}$上的标准距离定义为$d(x, y) = |x - y|$. 这使得$\mathbb{R}$成为一个距离空间.
\end{example}

\begin{example}
    欧几里得空间$\mathbb{R}^n$上的标准距离定义为$d(\mathbf{x}, \mathbf{y}) = \|\mathbf{x} - \mathbf{y}\|_2 = \sqrt{\sum_{i=1}^n (x_i - y_i)^2}$. 这使得$\mathbb{R}^n$成为一个距离空间.
\end{example}

\backmatter


\end{document}
